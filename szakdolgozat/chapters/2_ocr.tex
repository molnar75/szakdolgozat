\Chapter{Elektronikus dokumentumok}

Az elektronikus dokumentumok fokozatosan kezdik felváltani a papír alapú megfelelőiket.
Teljesen nyilvánvaló, könnyen hozzáférhető eszközök segítségével elvégezhető, hogy ilyen dokumentumokat megtekintsünk, létrehozzunk, szerkesszünk vagy valamilyen módon megosszunk.

A fejezetben bevezetésképpen azt tekintjük át, hogy milyen elterjedt és szabványosított dokumentum formátumok vannak.
A leírás célja ezen túlmenően az is, hogy megindokolja, hogy miért éppen a PDF formátumra esett a választás.

\Section{Átvihető dokumentum formátumok}

Az elektronikus dokumentumoknak különféle formátumai vannak.
A dolgozat szempontjából kiemelt fontosságú a PDF (\textit{Portable Document Format}), mivel az tekinthető a leginkább elfogadott, és mint a neve is mutatja, valószínűleg a legtöbb platformra átvihető dokumentum formátumnak.
Az aktuálisan használt specifikációját az \textit{Adobe} nevű cég kezeli \cite{pdf}.

Gyakran használt dokumentum formátum az RTF (\textit{Rich Text Format}), a DOC, a DOCX amelyet elsősorban a MicroSoft Office programcsomag tett népszerűvé. E mellett további nyílt dokumentumformátumokkal is találkozhatunk, mint például az ODT (\textit{Open Document Text}).

Elterjedtségét illetően a PDF vetélytársának a HTML (\textit{HyperText Markup Language}) nyelven írt dokumentumokat tekinthetjük, viszont céljukban és kezelési módjukban is jelentősen különböznek.

A konverzió HTML-ből PDF formátumba a megfelelő stílusbeállítások figyelembevételével mindig elvégezhető. A nehézséget a másik irány jelenti, mivel az sosem tud tökéletesen működni, mivel a HTML nyelv a PDF eszközkészletének jelentős részét nem tartalmazza.
Ehhez kapcsolódóan találhatunk kutatási eredményeket a szakirodalomban \cite{jiang2009converting}.

A PDF formátum elviekben megjelenítőeszköztől függetlenül nagyon hasonlóan jelenik meg. (Két megjelenítő eszközön a dokumentumok megjelenítésének azonosságát a hagyományos tekintetben vesszük, ezért szerepel a megfogalmazásban az, hogy nagyon hasonlóak lesznek.)
Ezt úgy képes elérni, hogy (szöveges dokumentum esetén) a szöveges tartalom mellett akár karakter vagy ékezetek szintén is tárolja, hogy a megjelenített alakzatok melyik részének hova kell majd kerülnie.
Ennek ilyen formában megvannak az előnyei, amelyek egyúttal a dolgozat célkitűzésére, és a választott módszerekre is választ igyekszik adni.
\begin{itemize}
\item A szövegelemek pontos helyének tárolása előnyös, mert így a szöveg tördelésével nem szükséges később már foglalkozni. Ez a megjelenítés sebessége és a dokumentum egysége szempontjából is lényeges. Azt mondhatjuk, hogy a dokumentum egy egységként tárolja a dokumentum helyes megjelenítéséhez szükséges adatok jelentős részét. Ez többek között azt is jelenti, hogy általában nem kell attól tartani, hogy egy adott betűtípus, vagy szimbólum a megjelenítő rendszeren nem lesz elérhető.
\item Mivel több adatról van szó, ezért hátránynak tekinthető, hogy maga a dokumentum mérete nagyobb lesz. Ez indokolatlanul nagy fájlméretet is jelenthet, mivel ha csak a dokumentumban tárolt lényegi információkra van szükségünk, akkor egy PDF fájl mérete több ezerszerese lehet az egyszerű szöveges tárolási módénak.
\item A PDF fájlokban a szövegek kijelölése időnként körülményes lehet. Gyakran találkozni például a sortörésekből adódó másolási hibákkal, a lemaradt ékezetekkel, a karakterek indokolatlan sorrendjével vagy akár nem látható, helytelen karakterek megjelenésével.
\item Attól függően, hogy a PDF milyen eszközzel készült, tartalmazhat elválasztásokat, speciális tördelési módokat.
\end{itemize}

\Section{A dokumentum tartalma}

A dokumentumokban lévő adatok egy része a tartalmat, másik része pedig a hozzá szükséges formázást írja le.
A dolgozat célja, hogy bemutasson olyan módszereket, amelyekkel ez a kettő elválasztható egymástól.
Az optikai karakterfelismerő rendszerek (OCR, \textit{Optical Character Recognition}) tulajdonképpen ezt a problémát oldják meg általánosan \cite{mori1999optical}.
A szakirodalomban ehhez kapcsolódóan teljes dokumentumok strukturális elemzésére is láthatunk példákat, ahol a dokumentum képének szegmentálása adja az elemzés alapját \cite{fujisawa1992segmentation}.

A PDF fájlformátum közismert, számos API is elérhető hozzá, viszont elég bonyolult (a benne rejlő számos lehetőség miatt).
Ahogy korábban említésre került, a karakterek elrendezése, tördelése elválaszthatja az egyébként egymáshoz tartozó adatokat, így a szövegblokkok kiolvasása önmagában nem elegendő.
Arra tehát nem tünt érdemesnek hagyatkozni, hogy a PDF fájlban lévő adatokat valamilyen függvénykönyvtár segítségével közvetlenül ki lehessen nyerni.
Helyette inkább az a feltételezés tünt megfelelőnek, hogy egy PDF dokumentum egy minimális mennyiségű zajjal terhelt bemeneti képnek tekinthető, amelyből adatokat a hagyományos képfeldolgozási módszerekkel ki lehet nyerni. 

\Section{Optikai karakterfelismerés}

Az optikai karakterfelismerési problémának számos változatai és előfordulási módjai vannak.
A módszereket például az alábbi szempontok alapján csoportosíthatjuk.
\begin{itemize}
\item Az elemzett szöveg alapján el szoktak különíteni géppel- és kézzel írott karakterekre kifejlesztett karakterfelismerő módszereket.
\item Jelentős különbségek vannak annak megfelelően, hogy mennyi szimbólumot tartalmaz az az ábécé, amelyből a szöveg karakterei kikerültek. Különbségek adódnak továbbá a szöveg elrendezésének változatosságából, a szövegképben előforduló hibákból, zajokból.
\item Az alkalmazott heurisztika alapján beszélhetünk determinisztikus és sztochasztikus módszerekről. Utóbbi olyan esetekben fordul elő gyakrabban, amikor a problémát egy optimalizálási feladatra vezetik vissza, és a keresési tér mérete miatt véletlenszerű, futtatásonként különböző eredményeket adhat az algoritmus.
\end{itemize}
A dolgozat arra törekszik, hogy a kifejezetten előnyösnek mondható képfeldolgozási problémát az elérhető eszközökkel minél egyszerűbben, és elegánsabban oldja meg.
A képfeldolgozással kapcsolatos problémákat az \textit{OpenCV} (\textit{Open Computer Vision}) nevű függvénykönyvtár segítségével oldottam meg \cite{opencv}.

\Section{Gépi tanulási módszerek}

A képfeldolgozási feladat is egy adatfeldolgozási probléma. A hagyományos módszertan szerint először egy előfeldolgozási, zajszűrési lépés szükséges, amit a jellegvektorok kinyerése követ majd.
Mivel a PDF fájlokban fehér zajra, megvilágítási hibákra, foltszerű zajokra nem kell számítanunk, ezért a feldolgozást gyakorlatilag kezdhetjük akár a jellegvektorok kinyerésével.

A Python-hoz szabadon elérhetők a \textit{Tesseract} \cite{patel2012optical} és a \textit{Keras} \cite{geron2019hands} függvénykönyvtárak, amelyeket előszeretettel használnak optikai karakterfelismerési problémák megoldására.
A gépi tanulás tanító- és tesztminták meglétét feltételezi. Mivel hosszabb terjedelmű PDF dokumentumokat igen könnyű találni, ismert szövegből is egyszerűen létre lehet őket hozni, így a tanító és tesztminták előállítása ezek segítségével megoldható.

A későbbiekben láthatjuk majd, hogy milyen módon és milyen eredményességgel lehet alkalmazni ezeket az aktuális problémára.
